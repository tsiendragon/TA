\documentclass[11pt]{beamer}
\usetheme{metropolis}           % Use metropolis theme
\metroset{numbering=fraction,background=light}
 
\let\oldsection\section
 \def\section#1{\metroset{numbering=fraction,background=dark}
 \oldsection{\arabic{section}. #1}\metroset{numbering=fraction,background=light}
}
%% ---------------------------------------showing comments --------------------------------------------------

%\usepackage{pgfpages}
%\setbeameroption{show notes}
%\setbeameroption{show notes on second screen=right}
%%-----------------------------------------------------------------------------------------------------------
\usepackage{xcolor}
\definecolor{color1}{HTML}{CD853F}
\definecolor{color2}{HTML}{1E90FF}
\definecolor{color3}{HTML}{FFFFFF}
\definecolor{color4}{HTML}{2F4F4F}
%%%%
\definecolor{foreground}{RGB}{255,255,255}
\definecolor{background}{RGB}{24,24,24}
\definecolor{title}{RGB}{107,174,214}
\definecolor{gray}{RGB}{155,155,155}
\definecolor{subtitle}{RGB}{102,255,204}
\definecolor{hilight}{RGB}{102,255,204}
\definecolor{vhilight}{RGB}{255,111,207}

%% -------------------- ------------------------------------------------------------
\setbeamercolor{progress bar}{ fg=color1, bg=color2 }
\setbeamercolor{title separator}{fg=color1,bg=color2 }
\setbeamercolor{progress bar in head/foot}{fg=color1,bg=color2 }
\setbeamercolor{progress bar in section page}{fg=color1,bg=color2}

%\setbeamercolor{titlelike}{fg=title}
%\setbeamercolor{subtitle}{fg=subtitle}
%\setbeamercolor{normal text}{fg=foreground,bg=background}
%\setbeamercolor{item}{fg=foreground} % color of bullets
%\setbeamercolor{subitem}{fg=gray}
%\setbeamercolor{itemize/enumerate subbody}{fg=gray}
\usepackage{tikz}

\usepackage{ragged2e} %% align with both side
\justifying
\let\raggedright\justifying\relax
\usepackage{unicode-math}  %% use for serif math font
\usepackage{amssymb}
\usepackage{amsmath}
\usepackage{booktabs}
%\usepackage{bm}

\usepackage{MnSymbol} %% diamond symbols for itemize 
%\usepackage{txfonts}  %% filled diamondsuit

\DeclareMathSizes{11}{11}{8}{6}
\usepackage{fontspec}
%\usepackage[utopia]{mathdesign}
%\usefonttheme[onlymath]{serif}
\usefonttheme{serif}
%\setsansfont[BoldFont={Sketch Rockwell}]{Myriad Roman}
\setmainfont[BoldFont={Sketch Rockwell}]{Myriad Roman}

%\setmonofont{PragmataPro}
\newfontfamily\DVS{RollandinEmilie}


\usepackage{float} % provide float option H  
\usepackage{url}
\usepackage{hyperref}
\hypersetup{
    colorlinks = true,
    linkbordercolor = {white},
    urlcolor={blue}
  }
\setlength{\floatsep}{3pt plus 1pt minus 2pt}
\setlength\intextsep{3pt plus 2pt minus 2pt}
\setlength\textfloatsep{3pt plus 2pt minus 2pt}
\addtobeamertemplate{block end}{}{\vspace*{-3pt}}



% logo of my university 
\titlegraphic{\vspace{10.5em}\hspace{13em}\includegraphics[width=4cm]{NUS}}


%% ------------------------------ newcommands --------------------
\usepackage{xparse}
\DeclareDocumentCommand{\MyIncludeGraphics}{ O{} +m }
{
    \IfFileExists{#2}
    {
        \includegraphics[#1]{#2}%
    }
    {
        \includegraphics[#1]{#2}%
    }
  }
 % \newcommand{\putpic}[3][5cm]{\mbox{}\only<#2>{ \begin{wrapfigure}[5]{r}{#1}
 %       \MyIncludeGraphics[width={#1}]{#3}  
 %   \end{wrapfigure}
 % }}
%\usepackage{showframe}
%\usepackage{twoopt}
\newlength\mylength
%\newcommandtwoopt{\putpic}[4][5cm][0.7cm]{
\DeclareDocumentCommand{\putpic}{ O{5cm} O{0.7cm} m m}{\setlength\mylength{\textwidth}\addtolength\mylength{-#1}\only<#3>{\hspace*{\mylength}\raisebox{-\height}[0pt][0pt]{\MyIncludeGraphics[width={#1}]{#4}}}}

      


\newcommand{\mya}[1]{\textcolor{red}{ {#1}} }


\newcommand{\phtitle}[2][-2em]{\hspace*{#1}\textbf{#2}}
\newcommand{\lenitem}[2][.58\linewidth]{\parbox[t]{#1}{\strut #2\strut}}

\newcommand{\putfig}[2][.4\linewidth]{
  \mbox{}\hfill\raisebox{-\height}[0pt][0pt]{
    \MyIncludeGraphics[width=#1]{#2}
  }\vspace*{-\baselineskip}}

\setbeamertemplate{itemize item}{\Large\raise1.25pt\hbox{\color{red}{\donotcoloroutermaths$\filleddiamond$}}}
\setbeamertemplate{itemize subitem}{\tiny\raise1.5pt\hbox{\color{red}{\donotcoloroutermaths$\blacktriangleright$}}}%
%\renewcommand{\labelitemi}{\color{blue}{\boxempty}}
%%
%%

\setbeamercolor{block title}{use=structure,fg=white,bg=gray!85!black}
\setbeamercolor{block body}{use=structure,fg=black,bg=gray!20!white}
\newtheorem{question}{Question}
\newcommand{\Mod}[1]{\ (\mathrm{mod}\ #1)}

\begin{document}
\title{GEH1036/GEK1505 Tutorial 9(week 11) }
\date{\today}
\author{Tsien Lilong}
\institute{Department of Mathematics,NUS}
\maketitle
\note{say "hello" now}
\begin{frame}{Personal Information}
  \begin{itemize}
    \item Tsien Lilong
          \begin{itemize}
            \item Email: qian.lilong@u.nus.edu
            \item Phone Number: 90874186
            \item Website: \underline{\url{tsien.farbox.com}}\\
                  Go to the website, and scroll down, see the \textbf{Work} entry, in the ``Tutor'' project, click the link
                  for more information. I will provide this slide and the source \textcolor{blue}{\it.tex} file on that site.
          \end{itemize}
    \item Course information
          \begin{itemize}
            \item EVERY WEEK	MONDAY	10:00-11:00	S16-0431
          \end{itemize}

          
  \end{itemize}
\end{frame}
\section{Encrypt the message}
\begin{frame}{Encrypt the message}
	\begin{question}
		Replace the letters in the following message by numbers according to the rule:  A = 0,B = 1,...,Z = 25.\\
\hfill
	BE BACK BY ELEVEN
\hfill\null\\
Hence encrypt the message using the shift transformation given by $y ≡ x+15 \Mod{26}$.
	\end{question}
	\begin{itemize}
		\item According to the coding rule, the message is converted to numbers as \\
		\begin{tabular}{*{14}{p{0.5em}} }
			01 & 04  &01  &00  &02  &10  &01  &24  &04 & 11 & 04  &21 & 04  &13\\
		\end{tabular}
	\item Then shift by $15\Mod{26}$,\\
	\begin{tabular}{*{14}{p{0.5em}}}
		16 & 19 & 16  &15  &17 & 25 & 16&  13 & 19 & 00 & 19 & 10 & 19&  01\\
	\end{tabular}
\item Back to the  ciphertext in letters: “QT QPRZ QN TAKTC”.
	\end{itemize}
\end{frame}


\section{Frequency analysis}
\begin{frame}{Frequency analysis}
	\begin{question}
		The following message encrypted by a shift transformation was received  "CXJFIFXOFQV YOBBAP ZLKQBJMQ".
Use frequency analysis to decipher it.
	\end{question}
\begin{itemize}
	\item The most frequently occurring letters are F, Q and B, with the same count 3.
	\item The letter used most frequently in English language is  E. Suppose the shift is k, i.e. $y=x+k$.
	\begin{itemize}
		\item If F is encoded by E, then $5=4+k\Mod{26}$. Then $k=1$. The deciphering transformation is thus $x=y-1$, which gives "BWIEHE$\cdots$". Does not make sense. 
		\item If Q is E. Then $16=4+k\Mod{26}$, i.e. $k=12$. Decipher it by $x=y-12 \Mod{26}$, which gives "QLXSWS$\cdots$". Does not make sense.
	\end{itemize}
\end{itemize}
\end{frame}
\begin{frame}
		\begin{question}
		The following message encrypted by a shift transformation was received  "CXJFIFXOFQV YOBBAP ZLKQBJMQ".
		Use frequency analysis to decipher it.
	\end{question}
\begin{itemize}
	\item If B is E, i.e., $1 =  4+k\Mod{26}$ and $k = −3$. The deciphering transformation is $x =y+3\Mod{26}$. This gives:
"FAMILIARITY BREEDS CONTEMPT".
\end{itemize}
\end{frame}

\section{Inverse of modulo}
\begin{frame}{Inverse of modulo}
	\begin{question}
		For each value of $a$ between 1 and 26 inclusive, use trial and error to find the inverse of a modulo 27, if it exists, i.e., find an integer $x$ between 1 and 26 inclusive such that $ax \equiv 1 (\mathrm{mod} 27)$.
	\end{question}
\begin{itemize}
	\item Start from $a=1$,  that is $x\equiv 1\Mod{27}$, then $x=1$.
	\item As for $a=2$, we have $2x\equiv 1\Mod{27}$, then $x={1+27k\over 2},k=0,1,\ldots$.  We get $x=14$.
	\item $a=3$	, we have $3x\equiv 1\Mod{27}$, then $x={1+27k\over 3},k=0,1,\ldots$, which has no integer solution. i.e. 
	Multiples of 3 don’t have inverses modulo 27
	\item similarly, the rest inverse of $a$ is given as \\
	$(a,x)=(4, 7), (5, 11), (7, 4), (8, 17), (10, 19), (13, 25), (16, 22),$\\\qquad\qquad$ (20, 23), (26, 26)$.
\end{itemize}
\end{frame}



\section{Enciphering function}
\begin{frame}{Enciphering function}
	\begin{question}
		The enciphering function on 27 symbols is given by the transformation $y = f(x) \equiv ax+b \Mod{27}$. Suppose that $f(1) = 2, f(2) = 6$, find the values of$ a$ and $b$. Hence find the deciphering function in the form
$x ≡ a′y + b′ \Mod{27},$ where$ a′,b′$ are integers between 1 and 26 inclusive.
	\end{question}
\begin{itemize}
	\item Find enciphering function. It is in fact to solve a linear system.
	By $f(1)=2,f(2)=6$, we have 
	\begin{align*}
	a+b&\equiv 2\Mod{27}\\
	2a+b&\equiv 6\Mod{27}
	\end{align*}
	Subtracting the first from the second, we get
$a ≡ 4$.
And $a+b\equiv 2\Mod{27}$ gives that 
$ b\equiv -2$.
\end{itemize}
\end{frame}
\begin{frame}
	\begin{itemize}
		\item Deciphering function is 
		\begin{align*}
		x\equiv a'y+b'\Mod{27}.
		\end{align*}
		Already we have 
		\begin{align*}
		y\equiv 4x-2\Mod{27}
		\end{align*}
		then 
		\begin{align*}
		4x\equiv y+2\Mod{27}
		\end{align*}
		And 
		\begin{align*}
		&7\times 4 x\equiv 7y+14\Mod{27}.\\
		\Rightarrow &27x+x\equiv 7y+14\Mod{27}
		\end{align*}
		Hence 
		\begin{align*}
		x\equiv 7y+14\Mod{27}.
		\end{align*}
	\end{itemize}
\end{frame}

\section{cyphering guess}
\begin{frame}{Cyphering guess}
	\begin{question}
		A message in English contains only letters of the alphabet and is encrypted using an affine transformation (with the original spacing words left intact). It is guessed that the letters E,T in the original message have been substituted by I,N respectively. It is also found that the sequences “JI” and “EJI” occur in isolated blocks many times in the encrypted message. Do you think that the guess above is correct? Justify your answer.
	\end{question}
\begin{itemize}
	\item Let the affine transformation be $y = f(x) ≡\equiv ax+b \Mod{26}$. 
	\item The guess is $f(4) = 8$ and $f(19) = 13$. Hence 
	\begin{align*}
	4a+b \equiv  8, &&19a+b \equiv 13.
	\end{align*}
\end{itemize}
\end{frame}
\begin{frame}
	\begin{itemize}
		\item Subtract the two congruence equations, we have $15a\equiv 5$, i.e. $15a-5=5(3a-1)=26k$. Note that $26$ is not divisible by $5$, hence $26$ must be divisible by  	$3a-1$. Then $a={1+26k\over 3},k=0,1,2,\ldots,$, which gives $a=9$.
		\item Then $b\equiv 8-4\times 9\equiv -2\Mod{26}$. 
		\item Thus we have 
		\begin{align*}
		y\equiv 9x-2
		\Rightarrow 9x\equiv y+2
		\Rightarrow x\equiv 3y+6
		\end{align*}
		The last equation is obtained for the fact the inverse of 9 mod 26 is 3. 
		\item Using this deciphering formula,  note that $E=4,I=8,J=9$,
		\begin{align*}
		y=4, x≡3×4+6≡18=S \\
		y=8, x≡3×8+6≡4=E\\
		 y=9, x≡3×9+6≡7=H
		\end{align*}
		\item Thus “JI” and “EJI” are “HE” and “SHE” respectively. It is likely that the guess is correct.
	\end{itemize}

\end{frame}

\section{Modulo exponentiation}
\begin{frame}{Modulo exponentiation}
	\begin{question}
		 Find the remainder when 1243 is divided by 713 using modular exponentiation.
	\end{question}
\begin{itemize}
	\item Write exponent as sum of powers of 2: $43 = 2^5 +2^3 +2+1 = 32+8+2+1$.
	\item Compute 12 raised to powers which are powers of 2 mod 713:
	\begin{align*}
	12^2\equiv 144\\
	12^4\equiv 144^2\equiv 59\\
	12^8\equiv 59^2\equiv 629\\
	12^16\equiv 629^2\equiv 639\\
	12^32\equiv 639^2\equiv 458
	\end{align*}
\end{itemize}
\end{frame}
\begin{frame}{Modulo exponentiation}
	\begin{question}
		Find the remainder when 1243 is divided by 713 using modular exponentiation.
	\end{question}
	\begin{itemize}
		\item Finally
		$$ \hspace*{-3em}12^{43} =12^{13}\times 12^8\times 12^2\times 12^1\equiv 458\times 629\times 144\times 12 \equiv 48\Mod{713}$$
	\end{itemize}
\end{frame}


\section{Inverse for 43 modulo 600}
\begin{frame}{Inverse for 43 modulo 600}
	\begin{question}
		Find an inverse for 43 modulo 600 that lies between 1 and 600, i.e., find an integer $1\leqslant t\leqslant 600$ such that $43t\equiv1 \Mod{600}$.
	\end{question}
	\begin{itemize}
		\item For such question, first  apply Euclidean algorithm, the write the GCD (1 in fact) in terms of  two numbers.
		\item Apply Euclidean algorithm:
		\begin{align*}
		600 &= 43·15+15\\
43 &= 15·2+13 \\
15 &= 13·1+2 \\
13 &= 2 · 6 + 1
		\end{align*}
	\end{itemize}
\end{frame}
\begin{frame}{Inverse for 43 modulo 600}
	\begin{itemize}
		\item Then work backwards:
		\begin{align*}
		1 &= 13 − 2 · 6\\
&= 13−(15−13·1)6 = 13·7−15·6\\
&= (43−15·2)7−15·6 = 43·7−15·20\\
&= 43·7−(600−43·15)20 = 43·307−600·20
		\end{align*}
		\item Taking modulo 600, we have $1\equiv 43\cdot 307\Mod{600}$.
		\item Thus $t=307$ is an inverse.
	\end{itemize}
\end{frame}

\section{RSA}
\begin{frame}{RSA}
	\begin{question}
		Consider the RSA cryptosystem with $ p = 11, q = 17$, so that $n = pq = 187$ and with $k = 7$.\\
(a) Encrypt the message HI.\\
(b) Decrypt the encrypted message found in (a).\\
	\end{question}
\begin{itemize}
	\item H is 08 and I is 09.
	\item Encrypt each  of H,I. 
	\begin{itemize}
		\item 	Compute the remainder of $M^k$ on division by n. $8^7\equiv 134\Mod{187}$ and $9^7\equiv 70\Mod{187}$.
		\item The ciphertext is the string of digits of the remainders: 134 70.
	\end{itemize}

\end{itemize}
\end{frame}
\begin{frame}{RSA}
	To decrypt the message 134 70.
	\begin{itemize}
		\item $(p-1)(q-1)=160$.
		\item Find a inverse  $j$ of $k$ mod 160.  Using Euclidean algorithm
		\begin{align*}
		160 &= 7·22+6 7=6·1+1\\
		 1&=7−6·1= 7−(160−7·22)\\
& = 7·23−160·1
		\end{align*}
		Hence the $j=23$.
		\item To decrypt:
		\begin{align*}
			134^j\equiv134^{23}\equiv 8 && 70^{j}\equiv 70^{23}\equiv 9\Mod{187}.
		\end{align*}
		\item Thus the message is HI.
	\end{itemize}

\end{frame}
\let\oldframetitle\frametitle\def\frametitle#1{\oldframetitle{\arabic{section}. #1} } 
\begin{frame}[standout]
 \begin{center}
      \bf\LARGE\color{red} \underline{Thank you for listening!}
    \end{center}
  
\end{frame}
\end{document}
%%% Local Variables:
%%% mode: latex
%%% TeX-master: t
%%% End:
