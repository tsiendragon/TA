\documentclass{article}
\input{../../tsien/package-article.tex}
 \geometry{
 a4paper,
 total={210mm,297mm},
 top=1.25in,
 bottom=1.25in,
 bindingoffset=0.2in,
 marginparwidth=1.5in,
 marginparsep=2em,
% textwidth=6in,
 left=1in,
  right=2in
%, showframe
}
%\usepackage{unixode}
%-------------------------------------------------- font --------------------------------------------------
%\usepackage{mtpro2}
\usepackage{fontspec,unicode-math}
\setmathfont{STIX Two Math}
\setmainfont{STIX Two Text}
\newfontfamily\ft{Sketch Rockwell}
\newfontfamily\fn{Edwardian Script ITC}
\newfontfamily\fu{URW Chancery L}
\newfontfamily\fm{YanTi}
%-------------------------------------------------- font --------------------------------------------------
\definecolor{color1}{HTML}{3c6748}
\definecolor{colorname}{HTML}{DE1B1B}
\definecolor{colorcite}{HTML}{67BCDB}
\definecolor{blue1}{RGB}{235,122,82}
\definecolor{blue2}{RGB}{215,112,92}
\definecolor{veryblue}{RGB}{128,62,62}
\definecolor{secnum}{HTML}{558C89}
\definecolor{ptctitle}{HTML}{F5F3EE}
\definecolor{ptcbackground}{HTML}{ECECEA}
\definecolor{colorpart}{HTML}{4A96AD}
\definecolor{colorsection}{HTML}{2F4F4F}

%-------------------------------------------------- using tikz for drawing --------------------------------
\usetikzlibrary{shapes.geometric}
%%%%%%%%%%%%%%%%%%%%%%%%%%%%%%%%%%%%%%%%%%%%%%%%%%%%%%%%%%%%%%%%%%%%%%%%%%%%
%% header
%%
\hypersetup{colorlinks=true,urlcolor=blue}

\newlength{\leftm}
\setlength{\leftm}{1in}
\addtolength{\leftm}{\hoffset}
\fancypagestyle{basicstyle}
{%
  \fancyhf{}
    \renewcommand{\headrulewidth}{0pt}
  \fancyhead[EC]{
    \begin{tikzpicture}[remember picture,overlay]
      \coordinate (A) at (current page.north west);
      \coordinate (B) at (current page.north east);
      \coordinate (C) at ([xshift = -0.5\textwidth-\leftm]0,\voffset);
      \coordinate (D) at ([xshift = 0.5\textwidth]0,\voffset);      
      \node[diamond,fill=blue1,draw=blue2,font=\small,left] at (C)  {  \color{white}  \thepage};
      \draw[color = blue1,line width = 1.2] (C) -- (D);
      \draw[color = blue1,line width = 1] ([yshift = 1.5pt,xshift = -1.5pt]C)  -- ([yshift = 1.5pt,xshift = -25pt]D);
      \node[above,color = color1] at (0,\voffset) {\fm\rightmark};
  \end{tikzpicture}
  }
  \vspace{14mm}
  \fancyhead[OC]{
     \begin{tikzpicture}[remember picture,overlay]
      \coordinate (A) at (current page.north west);
      \coordinate (B) at (current page.north east);
      \coordinate (C) at ([xshift = 0.5\textwidth+\leftm]0,\voffset);
      \coordinate (D) at ([xshift = -0.5\textwidth]0,\voffset);
      \node[diamond,fill=blue1,draw=blue2,font=\small,right] at (C)  {  \color{white}  \thepage};
      \draw[color = blue1,line width = 1.2] (C) -- (D);
      \draw[color = blue1,line width = 1] ([yshift = 1.5pt,xshift = 1.5pt]C)  -- ([yshift = 1.5pt,xshift=-\textwidth - 0.6\leftm]C);
      \node[above, color=blue2] at (0,\voffset) {\ft \leftmark};
  \end{tikzpicture}
  }

  \vspace{14mm}


}

%% ann mark: marginal remarks
\newcommand{\annmark}[1]{%
  \textcolor{red}{$\longrightarrow\bm$#1$\bm $}%
}%

\newcommand{\ann}[1]{%
    \begin{tikzpicture}[remember picture, baseline=-0.75ex]%
        \node[coordinate] (inText) {};%
    \end{tikzpicture}%
    \marginpar{%
        \renewcommand{\baselinestretch}{1.0}%
        \begin{tikzpicture}[remember picture]%
            \definecolor{orange}{rgb}{1,0.5,0}%
            \draw node[fill=red!20,rounded corners,text width=\marginparwidth] (inNote){\footnotesize#1};%
    \end{tikzpicture}%
    }%
    \begin{tikzpicture}[remember picture, overlay]%
        \draw[draw = orange, thick]
            ([yshift=-0.2cm] inText)
                -| ([xshift=-0.2cm] inNote.west)
                -| (inNote.west);%
    \end{tikzpicture}%
}%
%%--------------------insert programming code
\usepackage{listings}
\definecolor{mygreen}{rgb}{0,0.6,0}
\definecolor{mygray}{rgb}{0.8,0.8,0.8}
\definecolor{mymauve}{rgb}{0.58,0,0.82}
\lstset{ %
  backgroundcolor=\color{mygray},   % choose the background color; you must add \usepackage{color} or \usepackage{xcolor}
  basicstyle=\scriptsize\ttfamily,        % the size of the fonts that are used for the code
  breakatwhitespace=false,         % sets if automatic breaks should only happen at whitespace
  breaklines=true,                 % sets automatic line breaking
  captionpos=t,                    % sets the caption-position to bottom
  commentstyle=\color{mygreen},    % comment style
  deletekeywords={...},            % if you want to delete keywords from the given language
  escapeinside={\%*}{*)},          % if you want to add LaTeX within your code
  extendedchars=true,              % lets you use non-ASCII characters; for 8-bits encodings only, does not work with UTF-8
  frame=single,                    % adds a frame around the code
  keepspaces=true,                 % keeps spaces in text, useful for keeping indentation of code (possibly needs columns=flexible)
  keywordstyle=\color{blue},       % keyword style
  language=Matlab,                 % the language of the code
  otherkeywords={*,...},            % if you want to add more keywords to the set
  numbers=left,                    % where to put the line-numbers; possible values are (none, left, right)
  numbersep=5pt,                   % how far the line-numbers are from the code
  numberstyle=\tiny\color{gray}, % the style that is used for the line-numbers
  rulecolor=\color{black},         % if not set, the frame-color may be changed on line-breaks within not-black text (e.g. comments (green here))
  showspaces=false,                % show spaces everywhere adding particular underscores; it overrides 'showstringspaces'
  showstringspaces=false,          % underline spaces within strings only
  showtabs=false,                  % show tabs within strings adding particular underscores
  stepnumber=1,                    % the step between two line-numbers. If it's 1, each line will be numbered
  stringstyle=\color{mymauve},     % string literal style
  tabsize=2,                       % sets default tabsize to 2 spaces
  title=\lstname,                   % show the filename of files
                                % included with \lstinputlisting; also
                                % try caption instead of title
framextopmargin=5pt
}
\usepackage{matlab-prettifier}
\usepackage{tabularx}
% -------------------------------------------------- new theorem  --------------------------------------------------
\newtheorem{defi}{Definition}
\newtheorem{remark}{Remark}
\newtheorem{thm}{Theorem}
\newtheorem{corollary}[thm]{Corollary}
\newtheorem{question}{Question}
\newtheorem{proposition}[thm]{Proposition}
\newtheorem{lem}[thm]{Lemma}
\newtheorem{ex}{Exercise}
%% -------------------------------------------------- newcommand --------------------------------------------------

%% -------------------- symbols same
\newcommand{\real}{\mathbb{R}}
\newcommand{\complex}{\mathbb{C}}
\newcommand{\rank}[1]{\mathrm{rank} (#1)}
\newcommand{\range}[1]{\mathcal{R} (#1)}
\newcommand{\norm}[1]{\left\lVert#1\right\rVert}
\newcommand{\kernel}[1]{\mathrm{Ker} (#1)}
%\DeclareMathOperator{\trace}{Tr}

\newcommand{\ha}{\mathcal{H}_A}
\newcommand{\hb}{\mathcal{H}_B}
\newcommand{\hh}{\mathcal{H}}

\newcommand{\cnx}[1][n]{\mathbb{C}_{#1}[x]}
%%
%%-------------------- different notations
\def\symbtype{1}
\ifcase\symbtype%
%% default case
\newcommand{\adj}[1]{{#1}^{*}}
\newcommand{\conj}[1]{\overline{#1}}
\usepackage{mathtools}
\DeclarePairedDelimiterX\bra[1]{}{}{\adj{#1}}
\DeclarePairedDelimiter\ket{}{}
\DeclarePairedDelimiterX\braket[2]{}{}{(#1, #2)}
\DeclarePairedDelimiterX\mket[2]{}{}{#1\otimes#2}
\newcommand{\ei}[1][i]{e_{#1}} % or e_i
\newcommand{\ej}{e_j}

%%
%%
\or%
\newcommand{\adj}[1]{{#1}^{\dagger}}
\newcommand{\conj}[1]{{#1}^{*}}
\usepackage{mathtools}
\DeclarePairedDelimiter\bra{\langle}{\rvert}
%\DeclarePairedDelimiterX\ket{\lvert}{\rangle}
%\DeclarePairedDelimiterX\mket[2]{\lvert}{\rangle}{#1,#2}
%\DeclarePairedDelimiterX\braket[2]{\langle}{\rangle}{#1 \delimsize\vert #2}

\newcommand{\ei}[1][i]{\ket{#1}} % or e_i
\newcommand{\ej}{\ket{j}}

\fi

\newcommand{\xx}[1]{\adj{#1}#1}

%% -------------------------------------------------- qed --------------------------------------------------
\renewcommand{\qedsymbol}{$\blacksquare$}
%% -------------------------------------------------- end --------------------------------------------------


% \DeclareMathOperator{\arg}{arg}
\newcommand{\spn}[1]{\mathrm{span}\{#1\}}

%
\title{Theorem of Separability}
\author{Tsien Dragon}
\begin{document}

%%%%%%%%%%%%%%%%%%%%%%%%%%%%%%%%%%%%%%%%%%%%%%%%%%%
%%                 title page                            
%%
%%%%%%%%%%%%%%%%%%%%%%%%%%%%%%%%%%%%%%%%%%%%%%%%%%%
\begin{titlepage}
  \thispagestyle{empty}
\begin{tikzpicture}[remember picture,overlay]
% draw image
\node[inner sep=0] at (current page.center)
{\includegraphics[width=\paperwidth,height=\paperheight]{{/home/tsien/w/tex/img/cover2}}};
\node[rotate = 45] at (9,-5.5) {\Huge\ft\textcolor{color1}{ Tutorial of MATLAB}};
\node[rotate = 45, color=colorname] at (11.7,-11.1) {\Huge\bf \fn Tsien Dragon};
\node[rotate = 45, color = colorname] at (14,-13) {\bf \fu  National University of Singapore};
\node[rotate = 45, color = colorname] at (15,-11) {\bf \fu  Department of Mathematics};
\node[rotate = 45, color = white] at (15,-14.7) {\bf \ft  \today};
\node[rotate = 45, color = white] at (14.4,-17.6) {\ft qian.lilong@u.nus.edu};
\end{tikzpicture}
\clearpage
\end{titlepage}


	\tableofcontents

%
\newpage
\section{Introduction to MATLAB}
  \begin{itemize}
     \item MATLAB: Matrix Laboratory
    \item multi-paradigm numerical computing system and proprietary programming language 
    \item Object-oriented programming and procedure languages
    \item Developed by MathWorks Inc. in USA
    \item Alternatives: Octave
    \item Nus student software:    {
          \url{https://nusit.nus.edu.sg/services/software_and_os/software/software-student/\#install-matlab}}
    \item 
  Octave:{ \url{https://www.gnu.org/software/octave/}}
  \end{itemize}
  \bigskip

%

\section{Starting the software}
 
 
    	 Login with your NUS Net ID:{\color{blue} NUSSTU$\backslash$******* } and corresponding password
    
  \begin{figure}[H]
  	\centering
  	\includegraphics[width=0.7\linewidth]{Udemy-MATLAB-from-Beginner-to-Expert}
  	\caption{MATLAB Icon}
  	\label{fig:udemy-matlab-from-beginner-to-expert}
  \end{figure}
  	 Double click the MATLAB icon 



\begin{figure}[H]
	\centering
	\includegraphics[width=0.7\linewidth]{Figure3}
	\caption{MATLAB Interface}
	\label{fig:figure3}
\end{figure}





\section{Basic statements}

  \begin{itemize}
    \item Single statements
    \item assign to a variable
    \item Usage of semicolon"{\color{blue}:}"
    \item variable "{\color{blue}ans}"
    \item Comments "{\color{blue} \emph\%}"
    \item Rules for name variable: start by a letter, including letters, numbers and underscore "\_". Case sensitive
  \end{itemize}
  \begin{lstlisting}
 >> 1+2
>> a = 1+2
>> 1+2;
>> a = 1+2;
>> % a = 1+2;
>> A = 1+2;
>> 1A = 1+2;
>> A1 = 1+2;
>> A_1 = 1+2;
  \end{lstlisting}

\section{Variable}
%\subsection{real number and complex number}
\subsection{real number and complex number}
\begin{itemize}
  \item real number :
          1,1.1,1.1e+1($1.1\times 10^{1}$),1.1e-1($1.1\times 10^{-1}$),pi($\pi=$3.1415...)\\
  \item Complex number\\
        Default imaginary unit $i$ or $j$, which is $\sqrt{-1}$
          1+i,1-i,1+j,1-j,(1+j)'
\end{itemize}
 For operation of complex number:{\tiny \url{https://www.mathworks.com/help/matlab/complex-numbers.html}}


%\subsection{Vector and matrix}
\subsection{vector and matrix}
 
      \begin{itemize}
        \item   Row vector:  a,b
              
        \item Column vector: c,d,e
              
        \item Matrix: A,B,C ({\color{red} \small \tt Vector can be regarded as a special matrix})
        \item Special matrix: 0,I,1
      \end{itemize}
\begin{lstlisting}
    a = [1,2,3];
    b = [1 2 3];
    c = [1;2;3];d = a';
    e = transpose(b);
    A = [1 2;3 4];
    B = [1,2,3;4,5,6];
    C = B';
    O = zeros(4,3);
    I = eye(4,4);
    one = ones(5,5);
\end{lstlisting}
   
%\subsection{Special variable}
\subsection{Special variable}
  \begin{table}[H]
    \centering
    \tiny
    \begin{tabular}{ccccccccc}\toprule
      symbol &pi&1&0&true\\\midrule
      meaning& $\pi=3.14\ldots$&default double 1 or "true"& default 0 or "false"& logical 1\\\midrule
     symbol &false& inf& -inf & NaN\\\midrule
     meaning&logical 0 &$\infty$&$-\infty$&non a number:$\frac{0}{0}$\\\bottomrule
    \end{tabular}
  \end{table}

\section{Operators}
%\subsection{Arithmetic Operators}
\subsection{Arithmetic Operators}
  \begin{table}[H]
    \centering
    \begin{tabular}{ccccccc}\toprule
    Symbol & +&-&*& $/$&$\backslash$&\^{}\\\midrule
     Example&1+2&1-2&1*2&$1/2$&1$\backslash$ 2&2\^{} 2\\\midrule
     Result&2&-1&2&0.5&2&4\\\bottomrule
   \end{tabular}
   \caption{Arithmetic Operators}
   \end{table}
   
Refs. {  \url{https://www.mathworks.com/help/matlab/matlab_prog/matlab-operators-and-special-characters.html}}


%\subsection{Relation operators}
\subsection{Relation operators}
   \begin{table}[H]
     \small
     \centering
   \begin{tabular}{ccccccc}\toprule
    Symbol & ==&$\sim$ =&>& >=&<&<=\\\midrule
     Example&1==2&1$\sim$=2&2>2&2>=2& 2<2 &2<=2\\\midrule
     Result&0&1&0&1&0&1\\\bottomrule
   \end{tabular}
   \caption{Relation Operators}
 \end{table}
 Note that here ``1'' is of the logical type, means ``true'' and ``0'' means the logical value``false''.\\
 See the detail of variables by ``{\color{blue}who var}''.

%\subsection{Logical operators}
\subsection{Logical operators}
  \begin{table}[H]
    \centering
    \begin{tabular}{cccc}\toprule
      symbol & |&\& &-\\
      meaning & Or & And& Not\\
      Ex& 1|0& 1\& 0 &$\sim 0$ \\
      Result& 1& 0& 1\\
      Equal exp.& or(1,0)&and(1,0)&not(0)\\
      Another exp.& 1||0& 1\&\& 0& $\sim 0$\\\bottomrule
    \end{tabular}
  \end{table}

\section{Some commands}
  \begin{table}[H]
    \centering
    \begin{tabularx}{\linewidth}{lXXXXX}\toprule
      cmd: & clc & clear a & clear all & 1:3&1:2:3\\\midrule
      Result& clear screen & clear variable a& remove all variables & row vector [1,2,3]& row vector [1,3]\\\midrule
      cmd:& who a & whos &clf&help cmd& doc cmd\\\midrule
      Results: &see detail of variable a& see the details of all variables& clear the graph window& see help information
      of cmd& see document details of cmd\\\bottomrule
    \end{tabularx}
  \end{table}

\section{Math functoin}
\begin{table}
  \centering
  \scriptsize
  \begin{tabularx}{\linewidth}{ccccccccc}\toprule
    $abs(x)$&$sqrt(x)$& sign(x)& sin(x) & cos(x)& tan(x) & cot(x)& sec(x) & csc(x)\\
   
    |x|& $\sqrt{x}$ &signum function& sin(x) & cos(x) & tan(x)& cotangent of x& secant of x& The cosecant of x\\\bottomrule
  \end{tabularx}
  \caption{basic function}
\end{table}

\begin{table}
  \begin{tabularx}{\linewidth}{cccccc}\toprule
    $asin(x)$&$acos(x)$& atan(x) & acot(x)& asec(x) & acsc(x)\\
   
     sin$^{-1}$(x) & cos$^{-1}$(x) & tan$^{-1}$(x)& cot$^{-1}$(x)& sec$^{-1}$(x)&csc$^{-1}(x)$\\\bottomrule
   \end{tabularx}
   \caption{Inverse Trigonometric Functions}
   \end{table}
   \begin{table}
     \centering
   \begin{tabularx}{\linewidth}{ccccc}\toprule
   syntax& $exp(x)$&$log(x)$& log2(x) & log10(x)\\
   
     value&e$^{x}$ & log$_{e}(x)$ & log$_{2}$(x)& log$_{10}$(x)\\\bottomrule
   \end{tabularx}
   \caption{   Exponential and Logarithm Functions}
   \end{table}
 
 \section{Command window display output format}
   \begin{itemize}
     \item format short ( default): display 4 digits
           
     \item format long: display 15 digits
     \item format short e ( format shorte): scientific notation with 4 digits 
      \item            format long e: Short scientific notation with 15 digits
     \item format long g: scientific notation with a total of 15 digits for double values, and 7 digits for single values.
     \item format rat: Ratio of small integers.
     \item format compact:Suppress excess blank lines to show more output on a single screen.
     \item format loose:Add blank lines to make output more readable.
   \end{itemize}
 
 \section{matrix operation}
   \begin{itemize}
     \item Input matrix: A=[1,2,3;4,5,6]; or A(1,1)=1,$\ldots$A(2,3)=5;
     \item Get the size: [n1,n2] = size(A);
          {\small \color{blue} n1:row length, n2:  column length.}\\
           length(A)  gets the row length of A.
     \item Increase the matrix: 
           or A(1,4)=1;A(2,4) = 2;
     \item Matrix concatenation: row concatenation,A= [B,C] if column length equals. For example, B = [1;2];C=[3;4];\\
           column concatenation, A= [B;C] if row length equals. For ex. B = [1,2];C = [3,4];
   \end{itemize}
 
 \section{matrix indexing}
   A is a matrix of size m$\times n$
   \begin{itemize}
     \item A(i,j) : (i,j)-th entry of A
     \item A(i,:) : i-th row of A
     \item A(:,j): j-th column of A
     \item A(end,:): last row of A
           \item A(:,end-1); second last column of A
     \item A(a:b,c:d): submatrix of A from a to b row and c to d column.
           
     \item A(e,f)(e,f are two vectors): sub matrix of A row indexing in e, column indexing in f.\\
           For Ex.
           A = eye(5);e = [1,3,5];f=[3,4];A(e,f)
   \end{itemize}
   Note that index should not exceed the size of the matrix. A vector can be regarded as a matrix.
 
 \section{Matirx computation}
   \small
   
   \begin{tabularx}{\linewidth}{XX}\toprule
     \multicolumn{2}{c}{matrix  operation}\\\midrule
     A+B &matrix addition\\\midrule
A-B &matrix subtraction\\\midrule
t * A& scalar-matrix,$t\in R,C$\\\midrule
A * B& matrix multiplication\\\midrule
A\^{}n & A*A*A...*A, n times\\\midrule
A$\backslash$ B & inv(A)*B\\\midrule
A/B& B*inv(A)\\\bottomrule
\end{tabularx}
\\
\begin{tabularx}{\linewidth}{XX}\toprule
       \multicolumn{2}{c}{matrix  entrywise operation}\\\midrule
     A.+B &wrong expression\\\midrule
A.-B &wrong expression\\\midrule
t.*A&=t*A\\\midrule
A. * B& A(i,j)*B(i,j)\\\midrule
A.\^{}n & A(i,j).\^{} n\\\midrule
A.$\backslash$ B &$\frac{B(i,j)}{A(i,j)}$\\\midrule
A./B& $\frac{A(i,j)}{B(i,j)}$\\\bottomrule
   \end{tabularx}
 
 \section{related operation of matrix}
   \begin{itemize}
     \item A’: conjugate transpose of A
           \item A.': transpose of A
     \item det(A): determinant of a square matrix A
     \item rank(A): rank of a square matrix A
     \item eig(A): eigenvalues of a square matrix A
     \item inv(A):inverse of a square nonsingular matrix A
   \end{itemize}
 
 \section{Logical flow of programming}
 
 \section{Conditional statements}
      If statement
\begin{lstlisting}
a=1;b=2;
if a>b
           y=a;
else
           y=b;
end
\end{lstlisting}
 Find the largest number of \{a,b\}.
  \begin{lstlisting}
a=1;b=2;c=3;%
if a>b
   if a>c
     y=a;
   else
     y=c;
   end
elseif b>c
   y=b;
else
   y=c;
end
\end{lstlisting}
    Find the largest number of \{a,b,c\}.
                 
      switch statement
 

 
 \section{Loops}
   \begin{itemize}
     \item While loop:
\begin{lstlisting}
k=0;
y=0;
while k<10
k=k+1;
y=y+k;
end
\end{lstlisting}
           $y=\sum_{i=1}^{10}i$
     \item for loop
\begin{lstlisting}
y=0;
for  k=1:10
y=y+k;
end
\end{lstlisting}
           $y=\sum_{i=1}^{10}i$
   \end{itemize}
 

 \section{Termination of loops}
   \begin{itemize}
     \item {\tt\color{red} break} exits from the innermost loop
          
\begin{lstlisting}
k=0;
y=0;
while 1
  k=k+1;
  y=y+k;
  if k==10
     break
  end
end
\end{lstlisting}
           $y=\sum_{i=1}^{10}i$               
\begin{lstlisting}
A= [1,2,3;4,5,6];[n1,n2]=size(A);
y=0;
for i=1:100
  for j=1:100
    y = y+ A(i,j);
    if j==n2
      break
    end
  end
  if i==n1
    break
  end
end
\end{lstlisting}
           $y=\sum_{i,j}^{n1,n2}A_{ij}$               
             
   \end{itemize}
 
 

 \section{Termination of loops}
   \begin{itemize}
     \item  {\tt\color{red} continue} pass to the next loop without executing the following statements
     \item  {\tt\color{red} return}  exists  the scripts or function
   \end{itemize}
  
\begin{lstlisting}
y=0;
for  k=1:10
  if mod(k,2)==0 % if k is even 
    continue
    %% if condition holds, then pass to
    % next loop without executing the
    % following statements in the loop 
  end
    y=y+k;
end
\end{lstlisting}
     The sum of odd numbers from 1 to 10.
\begin{lstlisting}
A = randi(2,10,20)-1;
% create a matrix with 0 or 1
for i=1:10
  for j=1:20
     if A(i,j)==0;
      y=1; disp(A(i,j)); % display this variable
       return
     end
  end
end
\end{lstlisting}
     Check if A has a zero entry.
   
 
 
\end{document}

%%% Local Variables:
%%% mode: latex
%%% TeX-master: t
%%% End:
