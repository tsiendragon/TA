\documentclass[aspectratio=169]{beamer}
% other options: finnish, sectionpages

\usetheme{tut}

\usepackage[english]{babel}
\usepackage{csquotes}
\usepackage[style=authoryear,backend=biber]{biblatex}
\usepackage{filecontents}
\usepackage{booktabs}
\usepackage{blindtext}
\usepackage{pgfpages}
\usepackage{tabularx}
\hypersetup{colorlinks=true,urlcolor=blue}
%\setbeameroption{show notes on second screen}% you'll need pgfpages for this one and a suitable pdf viewer i.e. dspdfviewer
%\setbeameroption{show notes}

\usepackage{listings}

\lstset{
  language=matlab,
    basicstyle=\footnotesize, 
    showstringspaces=false,
  basicstyle=\scriptsize\ttfamily,
  backgroundcolor=\color{TUTGrey}
}

\begin{filecontents}{bibliography.bib}
@book{hoenig1997,
 author = {Hoenig, Alan},
 title = {TEX Unbound: Latex and TEX Strategies for Fonts, Graphics and More},
 year = {1997},
 publisher = {Oxford University Press, Inc.}
} 
@misc{tantau2013,
  title = {The BEAMER class -- user guide for version~3.36},
  author = {T.~Tantau, J.~Wright, V.~Mileti{\'c}},
  year = {2015},
}
@online{tutgraphic,
  title = {Graphic guidelines},
  author = {{Tampere University of Technology}},
  note = {tutka > image > printed publications > graphic guidelines}
}
\end{filecontents}
\addbibresource{bibliography.bib}

% The color palettes can be changed
%   primary: lower half of titlepage, lower half of left color beam, default block environment
%   secondary: upper half of left color beam, example block body
%   tertiary: top header showing outline
%   quaternary: example block title
%   structure: structure elements such as list items
%\setbeamercolor{palette primary}{fg=white,bg=TUTsecPetrol}
%\setbeamercolor{palette secondary}{fg=white,bg=TUTsecOrange}
%\setbeamercolor{palette quaternary}{use=palette secondary,bg=palette secondary.bg!50!black}
\title{MA2213, Numerical Analysis I, Laboratory 1}
\subtitle{Introduction to MATLAB }
\author{Qian Lilong}
\email{qian.lilong@u.nus.edu}
\institute{Department of Mathematics\\National University of Singapore}
\date{\today}

\begin{document}

\maketitle

\section*{Outline}
\begin{frame}{Outline}
	\tableofcontents
\end{frame}
%

\section{Introduction}
\begin{frame}{Introduction to MATLAB}
  \begin{itemize}
     \item MATLAB: Matrix Laboratory
    \item multi-paradigm numerical computing system and proprietary programming language 
    \item Object-oriented programming and procedure languages
    \item Developed by MathWorks Inc. in USA
    \item Alternatives: Octave
    \item Nus student software:  \\  {\tiny\rm
          \url{https://nusit.nus.edu.sg/services/software_and_os/software/software-student/\#install-matlab}}
    \item 
  Octave:\\{\tiny \url{https://www.gnu.org/software/octave/}}
  \end{itemize}
  \bigskip
\end{frame}
%

\section{Starting the software}
\begin{frame}{Starting, 1/2}
  \begin{columns}
  \begin{column}{0.475\textwidth}
    \begin{enumerate}
    	\item Login with your NUS Net ID:{\tt\color{blue} NUSSTU$\backslash$******* } and corresponding password
    	\item Double click the MATLAB icon 
    \end{enumerate}
  \end{column}
  \begin{column}{0.475\textwidth}
  \begin{figure}
  	\centering
  	\includegraphics[width=0.7\linewidth]{Udemy-MATLAB-from-Beginner-to-Expert}
  	\caption{MATLAB Icon}
  	\label{fig:udemy-matlab-from-beginner-to-expert}
  \end{figure}
  \end{column}
  \end{columns}
\end{frame}

\begin{frame}[containsverbatim]{Starting, 2/2}

\begin{figure}
	\centering
	\includegraphics[width=0.7\linewidth]{Figure3}
	\caption{MATLAB Interface}
	\label{fig:figure3}
\end{figure}
\end{frame}




\begin{frame}[containsverbatim]{Basic statements}

  \begin{itemize}
    \item Single statements
    \item assign to a variable
    \item Usage of semicolon"{\color{blue}:}"
    \item variable "{\color{blue}ans}"
    \item Comments "{\color{blue} \emph\%}"
    \item Rules for name variable: start by a letter, including letters, numbers and underscore "\_". Case sensitive
  \end{itemize}
  \begin{lstlisting}
 	>> 1+2
	>> a = 1+2
	>> 1+2;
	>> a = 1+2;
	>> % a = 1+2;
        >> A = 1+2;
        >> 1A = 1+2;
        >> A1 = 1+2;
        >> A_1 = 1+2;
  \end{lstlisting}
\end{frame}
\section{Variable}
%\subsection{real number and complex number}
\begin{frame}{real number and complex number}
\begin{itemize}
  \item real number :
        \begin{exampleblock}{Ex}
          1,1.1,1.1e+1($1.1\times 10^{1}$),1.1e-1($1.1\times 10^{-1}$),pi($\pi=$3.1415...)\\
        \end{exampleblock}
  \item Complex number\\
        Default imaginary unit $i$ or $j$, which is $\sqrt{-1}$
        \begin{exampleblock}{Ex}
          1+i,1-i,1+j,1-j,(1+j)'
        \end{exampleblock}
\end{itemize}
 For operation of complex number:{\tiny \url{https://www.mathworks.com/help/matlab/complex-numbers.html}}
\end{frame}

%\subsection{Vector and matrix}
\begin{frame}{vector and matrix}
  \begin{columns}
    \begin{column}{0.4\linewidth}
      \begin{itemize}
        \item   Row vector:  a,b
              
        \item Column vector: c,d,e
              
        \item Matrix: A,B,C ({\color{red} \small \tt Vector can be regarded as a special matrix})
        \item Special matrix: 0,I,1
      \end{itemize}
    \end{column}
    \begin{column}{0.4\linewidth}
      \begin{exampleblock}{Ex}
         a = [1,2,3];
    b = [ 1 2 3];\\
    c = [1;2;3];d = a';\\
    e = transpose(b);\\
    A = [1 2;3 4];\\
    B = [1,2,3;4,5,6];\\
    C = B';\\
    O = zeros(4,3);\\
    I = eye(4,4);\\
    one = ones(5,5);\\
      \end{exampleblock}
  \end{column}
\end{columns}
\end{frame}
%\subsection{Special variable}
\begin{frame}
  \frametitle{Special variable}
  \begin{table}
    \centering
    \tiny
    \begin{tabular}{ccccccccc}\toprule
      symbol &pi&1&0&true\\\midrule
      meaning& $\pi=3.14\ldots$&default double 1 or "true"& default 0 or "false"& logical 1\\\midrule
     symbol &false& inf& -inf & NaN\\\midrule
     meaning&logical 0 &$\infty$&$-\infty$&non a number:$\frac{0}{0}$\\\bottomrule
    \end{tabular}
  \end{table}
\end{frame}
\section{Operators}
%\subsection{Arithmetic Operators}
\begin{frame}
  \frametitle{Arithmetic Operators}
  \begin{table}
  \small
   \begin{tabular}{ccccccc}\toprule
    Symbol & +&-&*& $/$&$\backslash$&\^{}\\\midrule
     Example&1+2&1-2&1*2&$1/2$&1$\backslash$ 2&2\^{} 2\\\midrule
     Result&2&-1&2&0.5&2&4\\\bottomrule
   \end{tabular}
   \caption{Arithmetic Operators}
   \end{table}
   
Refs. {\tiny  \url{https://www.mathworks.com/help/matlab/matlab_prog/matlab-operators-and-special-characters.html}}
\end{frame}

%\subsection{Relation operators}
\begin{frame}
  \frametitle{Relation operators}
   \begin{table}
  \small
   \begin{tabular}{ccccccc}\toprule
    Symbol & ==&$\sim$ =&>& >=&<&<=\\\midrule
     Example&1==2&1$\sim$=2&2>2&2>=2& 2<2 &2<=2\\\midrule
     Result&0&1&0&1&0&1\\\bottomrule
   \end{tabular}
   \caption{Relation Operators}
 \end{table}
 Note that here ``1'' is of the logical type, means ``true'' and ``0'' means the logical value``false''.\\
 See the detail of variables by ``{\color{blue}who var}''.
\end{frame}
%\subsection{Logical operators}
\begin{frame}
  \frametitle{Logical operators}
  \begin{table}
    \centering
    \begin{tabular}{cccc}\toprule
      symbol & |&\& &-\\
      meaning & Or & And& Not\\
      Ex& 1|0& 1\& 0 &$\sim 0$ \\
      Result& 1& 0& 1\\
      Equal exp.& or(1,0)&and(1,0)&not(0)\\
      Another exp.& 1||0& 1\&\& 0& $\sim 0$\\\bottomrule
    \end{tabular}
  \end{table}
\end{frame}
\section{Some commands}
\begin{frame}{Some commands}
  \begin{table}
    \centering
    \tiny
    \begin{tabularx}{\linewidth}{lXXXXX}\toprule
      cmd: & clc & clear a & clear all & 1:3&1:2:3\\\midrule
      Result& clear screen & clear variable a& remove all variables & row vector [1,2,3]& row vector [1,3]\\\midrule
      cmd:& who a & whos &clf&help cmd& doc cmd\\\midrule
      Results: &see detail of variable a& see the details of all variables& clear the graph window& see help information
      of cmd& see document details of cmd\\\bottomrule
    \end{tabularx}
  \end{table}
\end{frame}
\section{Math functoin}
\begin{frame}{math function}
\begin{table}
  \centering
  \tiny
  \begin{tabularx}{\linewidth}{ccccccccc}\toprule
    $abs(x)$&$sqrt(x)$& sign(x)& sin(x) & cos(x)& tan(x) & cot(x)& sec(x) & csc(x)\\
   
    |x|& $\sqrt{x}$ &signum function& sin(x) & cos(x) & tan(x)& cotangent of x& secant of x& The cosecant of x\\\bottomrule
  \end{tabularx}
  \caption{basic function}
\end{table}

\begin{table}
  \begin{tabularx}{\linewidth}{cccccc}\toprule
    $asin(x)$&$acos(x)$& atan(x) & acot(x)& asec(x) & acsc(x)\\
   
     sin$^{-1}$(x) & cos$^{-1}$(x) & tan$^{-1}$(x)& cot$^{-1}$(x)& sec$^{-1}$(x)&csc$^{-1}(x)$\\\bottomrule
   \end{tabularx}
   \caption{Inverse Trigonometric Functions}
   \end{table}
   \begin{table}
     \centering
   \begin{tabularx}{\linewidth}{ccccc}\toprule
   syntax& $exp(x)$&$log(x)$& log2(x) & log10(x)\\
   
     value&e$^{x}$ & log$_{e}(x)$ & log$_{2}$(x)& log$_{10}$(x)\\\bottomrule
   \end{tabularx}
   \caption{   Exponential and Logarithm Functions}
   \end{table}
 \end{frame}
 \begin{frame}{Command window display output format}
   \begin{itemize}
     \item format short ( default): display 4 digits
           
     \item format long: display 15 digits
     \item format short e ( format shorte): scientific notation with 4 digits 
      \item            format long e: Short scientific notation with 15 digits
     \item format long g: scientific notation with a total of 15 digits for double values, and 7 digits for single values.
     \item format rat: Ratio of small integers.
     \item format compact:Suppress excess blank lines to show more output on a single screen.
     \item format loose:Add blank lines to make output more readable.
   \end{itemize}
 \end{frame}
 \section{matrix operation}
 \begin{frame}
   \frametitle{matrix operation}
   \begin{itemize}
     \item Input matrix: A=[1,2,3;4,5,6]; or A(1,1)=1,$\ldots$A(2,3)=5;
     \item Get the size: [n1,n2] = size(A);
          {\small \color{blue} n1:row length, n2:  column length.}\\
           length(A)  gets the row length of A.
     \item Increase the matrix: 
           or A(1,4)=1;A(2,4) = 2;
     \item Matrix concatenation: row concatenation,A= [B,C] if column length equals. For example, B = [1;2];C=[3;4];\\
           column concatenation, A= [B;C] if row length equals. For ex. B = [1,2];C = [3,4];
   \end{itemize}
 \end{frame}
 \begin{frame}{matrix indexing}
   A is a matrix of size m$\times n$
   \begin{itemize}
     \item A(i,j) : (i,j)-th entry of A
     \item A(i,:) : i-th row of A
     \item A(:,j): j-th column of A
     \item A(end,:): last row of A
           \item A(:,end-1); second last column of A
     \item A(a:b,c:d): submatrix of A from a to b row and c to d column.
           
     \item A(e,f)(e,f are two vectors): sub matrix of A row indexing in e, column indexing in f.\\
           For Ex.
           A = eye(5);e = [1,3,5];f=[3,4];A(e,f)
   \end{itemize}
   Note that index should not exceed the size of the matrix. A vector can be regarded as a matrix.
 \end{frame}
 \begin{frame}
   \frametitle{Matirx computation}
   \small
   
   \begin{tabularx}{0.48\linewidth}{cX}\toprule
     \multicolumn{2}{c}{matrix  operation}\\\midrule
     A+B &matrix addition\\\midrule
A-B &matrix subtraction\\\midrule
t * A& scalar-matrix,$t\in R,C$\\\midrule
A * B& matrix multiplication\\\midrule
A\^{}n & A*A*A..*A, n times\\\midrule
A$\backslash$ B & inv(A)*B\\\midrule
A/B& B*inv(A)\\\bottomrule
\end{tabularx}
\begin{tabularx}{0.48\linewidth}{cX}\toprule
       \multicolumn{2}{c}{matrix  entrywise operation}\\\midrule
     A.+B &wrong expression\\\midrule
A.-B &wrong expression\\\midrule
t. * A&=t*A\\\midrule
A. * B& A(i,j)*B(i,j)\\\midrule
A.\^{}n & A(i,j).\^{} n\\\midrule
A.$\backslash$ B &$\frac{B(i,j)}{A(i,j)}$\\\midrule
A./B& $\frac{A(i,j)}{B(i,j)}$\\\bottomrule
   \end{tabularx}
 \end{frame}
 \begin{frame}{related operation of matrix}
   \begin{itemize}
     \item A’: conjugate transpose of A
           \item A.': transpose of A
     \item det(A): determinant of a square matrix A
     \item rank(A): rank of a square matrix A
     \item eig(A): eigenvalues of a square matrix A
     \item inv(A):inverse of a square nonsingular matrix A
   \end{itemize}
 \end{frame}
 \section{Logical flow of programming}
 
 \begin{frame}[containsverbatim]{Conditional statements}
      If statement
\begin{columns}
\begin{column}{0.45\linewidth}
\begin{lstlisting}
a=1;b=2;
if a>b
           y=a;
else
           y=b;
end
\end{lstlisting}
 Find the largest number of \{a,b\}.
\end{column}
 \begin{column}{0.45\linewidth}
  \begin{lstlisting}
a=1;b=2;c=3;%
if a>b
   if a>c
     y=a;
   else
     y=c;
   end
elseif b>c
   y=b;
else
   y=c;
end
\end{lstlisting}
    Find the largest number of \{a,b,c\}.
         \end{column}
           \end{columns}
      switch statement
 \end{frame}

 
 \begin{frame}[containsverbatim]{Loops}
   \begin{itemize}
     \item While loop:
\begin{lstlisting}
k=0;
y=0;
while k<10
k=k+1;
y=y+k;
end
\end{lstlisting}
           $y=\sum_{i=1}^{10}i$
     \item for loop
\begin{lstlisting}
y=0;
for  k=1:10
y=y+k;
end
\end{lstlisting}
           $y=\sum_{i=1}^{10}i$
   \end{itemize}
 \end{frame}

 \begin{frame}[containsverbatim]{Termination of loops}
   \begin{itemize}
     \item {\tt\color{red} break} exits from the innermost loop
           \begin{columns}
             \begin{column}{0.45\linewidth}
\begin{lstlisting}
k=0;
y=0;
while 1
  k=k+1;
  y=y+k;
  if k==10
     break
  end
end
\end{lstlisting}
           $y=\sum_{i=1}^{10}i$               
             \end{column}
             \begin{column}{0.45\linewidth}
\begin{lstlisting}
A= [1,2,3;4,5,6];[n1,n2]=size(A);
y=0;
for i=1:100
  for j=1:100
    y = y+ A(i,j);
    if j==n2
      break
    end
  end
  if i==n1
    break
  end
end
\end{lstlisting}
           $y=\sum_{i,j}^{n1,n2}A_{ij}$               
         \end{column}
       \end{columns}
   \end{itemize}
 \end{frame}
 

 \begin{frame}[containsverbatim]{Termination of loops}
   \begin{itemize}
     \item  {\tt\color{red} break} exits from the innermost loop
     \item  {\tt\color{red} return}  exists  the scripts or function
   \end{itemize}
   \begin{columns}
     \begin{column}{0.5\linewidth}    
\begin{lstlisting}
y=0;
for  k=1:10
  if mod(k,2)==0 % if k is even 
    continue
    %% if condition holds, then pass to
    % next loop without executing the
    % following statements in the loop 
  end
    y=y+k;
end
\end{lstlisting}
     The sum of odd numbers from 1 to 10.
   \end{column}
   \begin{column}{0.5\linewidth}
\begin{lstlisting}
A = randi(2,10,20)-1;
% create a matrix with 0 or 1
for i=1:10
  for j=1:20
     if A(i,j)==0;
      y=1; disp(A(i,j)); % display this variable
       return
     end
  end
end
\end{lstlisting}
     Check if A has a zero entry.
   \end{column}
   \end{columns}
 \end{frame}
 
\end{document}

%%% Local Variables:
%%% mode: latex
%%% TeX-master: t
%%% End:
